\documentclass[12pt,a4paper]{article}

% ---
% PACOTES
% ---

% ---
% Pacotes fundamentais 
% ---
\usepackage{cmap}				% Mapear caracteres especiais no PDF
\usepackage{lmodern}			% Usa a fonte Latin Modern
\usepackage[T1]{fontenc}		% Selecao de codigos de fonte.
\usepackage[utf8]{inputenc}		% Codificacao do documento (conversão automática dos acentos) cada seção.
\usepackage{color}				% Controle das cores
\usepackage[fleqn]{amsmath}
\usepackage{graphicx}			% Inclusão de gráficos
\graphicspath{ {images/} }
% ---

\title{Problemas Frequentes na Instalação e execução dos projetos do IDEHCO3 e Lembretes}
\author{Aluizio Lima Filho}

\begin{document}
	
\maketitle

\section{Problemas de instalação}

\subsection{Erro na instalação do 'psycopg2'}
Pode ocorrer durante a instalação do psycopg2 um erro de compilção relacionado ao \verb|#include <Python.h>|, isso ocorre devido a falta do pacote python-dev.
Para solucionar este problema basta instalar o pacote python-dev,
para realizar tal tarefa basta executar o comando abaixo:
\begin{verbatim}
	$ sudo apt-get install python-dev
\end{verbatim}


\section{Problemas de execução}

\subsection{Erro ao executar o comando syncdb ou o comando migrate relacionado ao auth\_user}
Isso ocorre os o syncdb não consegue criar no banco a tabela dos usuários do django. Dessa forma podemos resolver isso com os dois comandos abaixo:

\begin{verbatim}
	$ python manage.py migrate auth
	$ python manage.py migrate
\end{verbatim}

\subsection{Erro ao executar o comando migrate e as tabelas não são criadas}
Isso pode ocorrer devido a pasta migrations que é criada pelo django dentro de cada aplicação, essa pasta de alguma forma inpede a criação das tabelas. Para resolver o problema basta excluir a pasta e executar o comando migrate novamente.

\section{Lembretes}

\subsection{Criação de um usuário no postgres}
Para criar um usuário no postgres é necessário seguir a seguinte sequência de passos:

\begin{enumerate}
	\item primeiro abrimos o terminal do linux.
	\item agora executamos o comando \verb|$ sudo su postgres| para mudar o usuário no terminal para postgres.
	\item em seguida executamos o comando \verb|$ psql| para entrar no programa responsável por executar commandos sql.
	\item e por último executamos o seguinte comando sql:
		\begin{verbatim}
			$ CREATE ROLE nomeDoUsuario 
			WITH SUPERUSER CREATEROLE CREATEDB LOGIN PASSWORD 'senha';
		\end{verbatim}
\end{enumerate}

Observação: lembre-se sempre de por o ponto e vírgula no final dos comandos sql, pois o psql só reconhece o fim do comando quando encontra o ponto e vírgula.

\subsection{Instalação do pip}
O pip é usado para instalação de pacotes python.
Para instalar o pip basta executar \verb|$ sudo apt-get install python-pip|.

\subsection{Uso do virtualenv}
Quando se está trabalhando com diferentes versões de um pacote python pode ser útil o uso do virtualenv.
Para instalar o virtualenv basta executar \verb|$ sudo pip install virtualenv|.
Para criar um ambiente virtual python é necessário executar o seguinte comando \verb|$ virtualenv nomeDoAmbiente|. Para entrar no ambiente virtual execute \verb|$ source nomeDoAmbiente/bin/activate|. A partir deste ponto qualuer pacote python que você instalar usando o comando pip será instalado nesse ambiente virtual.
\end{document}
